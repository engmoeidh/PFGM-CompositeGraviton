\documentclass[11pt]{article}

% ========== Packages ==========
\usepackage[a4paper,margin=1in]{geometry}
\usepackage{amsmath,amssymb,mathtools}
\usepackage{bm}
\usepackage{graphicx}
\usepackage{booktabs,tabularx,array}
\newcolumntype{L}[1]{>{\raggedright\arraybackslash}p{#1}}
\usepackage{siunitx}
\usepackage{hyperref}
\usepackage[nameinlink]{cleveref}
\usepackage{xcolor}
\usepackage{microtype}
\usepackage{enumitem}

% ========== Macros (minimal, extend as needed) ==========
\newcommand{\Geff}{g^{\mathrm{eff}}_{\mu\nu}}
\newcommand{\Geffinv}{g_{\mathrm{eff}}^{\mu\nu}}
\newcommand{\geff}{g_{\mathrm{eff}}}
\newcommand{\Mpl}{M_{\mathrm{Pl}}}
\newcommand{\Boxg}{\Box_{g_{\mathrm{eff}}}}
\newcommand{\PofX}{P(X)}
\newcommand{\dd}{\mathrm{d}}
\newcommand{\mpl}{M_{\mathrm{Pl}}}
\newcommand{\calO}{\mathcal{O}}

\crefname{section}{Sec.}{Secs.}
\crefname{appendix}{App.}{Apps.}
\crefname{figure}{Fig.}{Figs.}
\crefname{table}{Table}{Tables}

% ========== Title ==========
\title{Proto--Field Gravity V:\\
Emergent Spin--2 Sector and Composite Graviton}
\author{Moeidh M.~Hanash}
\date{\today}

\begin{document}
\maketitle

\begin{abstract}
We investigate the microscopic origin of the tensor sector in the proto--field
gravity model (PFGM), where spacetime geometry and its kinetic term emerge from
a single real scalar field with a rank--one disformal composite metric
\( g^{\rm eff}_{\mu\nu}[\Phi] = \eta_{\mu\nu} + \alpha\,\partial_\mu\Phi\,\partial_\nu\Phi \).
Building on previous work where an Einstein--Hilbert term and higher--curvature
corrections are induced by integrating out proto--field fluctuations, we ask
whether the microscopic Hilbert space actually contains a genuine spin--2
excitation that can be identified as a composite graviton.

On simple but nontrivial backgrounds with a constant proto--field gradient we
construct a composite metric fluctuation operator
\( h_{\mu\nu}[\Phi] \equiv \delta g^{\rm eff}_{\mu\nu}[\Phi] \) and derive
its two--point function from the quadratic proto--field action.  We show that
the corresponding momentum--space correlator admits a spin decomposition with a
massless spin--2 pole of positive residue in the transverse--traceless sector,
while scalar admixtures are either massive or parametrically suppressed in the
infrared.  In the induced--gravity window where the Einstein--Hilbert term
dominates over \(R^2\) and \(C^2\) operators, the linearised equations of motion
for this TT sector reduce to an Einstein--like wave equation sourced by the
standard stress--energy tensor, up to controlled higher--derivative corrections.

These results demonstrate that PFGM supports an emergent composite graviton
with effective dynamics indistinguishable from general relativity at long
wavelengths, providing a concrete microscopic realisation of a massless
spin--2 mode built from a single scalar proto--field.
\end{abstract}


\tableofcontents

% =========================================================
\section{Introduction}
\label{sec:intro}

The proto--field gravity model (PFGM) proposes that both spacetime geometry and
its dynamical kinetic term emerge from a single real scalar ``proto--field''
propagating on a fixed Minkowski background.  The effective geometry is
encoded in a rank--one disformal composite metric
\begin{equation}
  g^{\rm eff}_{\mu\nu}[\Phi] = \eta_{\mu\nu}
  + \alpha\,\partial_\mu\Phi\,\partial_\nu\Phi,
\end{equation}
and matter is minimally coupled to \(g^{\rm eff}_{\mu\nu}\).  Previous work has
shown that fluctuations of the proto--field can induce an Einstein--Hilbert
term and higher--curvature corrections for \(g^{\rm eff}_{\mu\nu}\), with
phenomenologically acceptable parameters and a viable weak--field limit.  In
particular, the PFGM framework has been tested against post--Newtonian
constraints, binary dynamics, and mesoscopic dark--matter phenomenology in a
series of companion papers.

A key conceptual question, however, remains open: what is the microscopic
origin of the tensor sector?  At the level of the induced gravitational
effective action, the theory contains an Einstein--Hilbert term and behaves
like general relativity in the infrared, but this does not by itself guarantee
that the underlying Hilbert space of the proto--field theory contains a
genuine massless spin--2 excitation.  In other words, we would like to know
whether there exists a composite operator built from \(\Phi\) whose
two--point function exhibits the characteristic tensor projector structure and
massless pole of a graviton, and whether this emergent spin--2 mode obeys
Einstein--like dynamics in the regime where the induced Einstein--Hilbert term
dominates.

The purpose of this paper is to address this microscopic question within the
PFGM setup.  Working on simple but nontrivial backgrounds with a constant
proto--field gradient, we construct a composite metric fluctuation operator
\(h_{\mu\nu}[\Phi]\equiv \delta g^{\rm eff}_{\mu\nu}[\Phi]\), derive its
two--point function from the quadratic proto--field action, and analyse its
spin content.  We show that the transverse--traceless sector of the
\(\langle h h\rangle\) correlator contains a massless spin--2 pole with
positive residue, while scalar admixtures are either massive or suppressed in
the infrared.  We then connect this composite operator to the induced
Einstein--Hilbert action and demonstrate that, in the appropriate low--energy
window, the linearised equations of motion for the transverse--traceless
sector reduce to an Einstein--like wave equation sourced by the standard
stress--energy tensor.

\subsection{Motivation and context}
\label{subsec:intro_motivation}

Emergent and induced gravity scenarios have long sought to explain the
Einstein--Hilbert action and its massless spin--2 excitations as effective
phenomena arising from more fundamental microscopic degrees of freedom.
Examples range from Sakharov--type induced gravity to analogue gravity systems
and various compositeness proposals for the graviton.  The PFGM framework adds
a particularly economical realisation: a single scalar field with a disformal
composite metric and a healthy higher--derivative microscopic action.  In
earlier work, the focus was on deriving the induced gravitational action,
establishing the existence of a viable Einstein--Hilbert term and constraining
higher--curvature operators using post--Newtonian and strong--field tests.

From this perspective, the present paper fills an important conceptual gap.  It
provides a concrete demonstration that the scalar proto--field can support an
emergent tensor sector whose long--wavelength dynamics are indistinguishable
from those of general relativity, and it clarifies how the effective
diffeomorphism invariance of the induced action is realised in terms of
underlying scalar degrees of freedom.

\subsection{Summary of results}
\label{subsec:intro_summary}

The main results of this paper can be summarised as follows:
\begin{itemize}[leftmargin=*]
  \item We construct a composite metric fluctuation operator
  \(h_{\mu\nu}[\Phi]\equiv \delta g^{\rm eff}_{\mu\nu}[\Phi]\) on backgrounds
  with a constant proto--field gradient and derive the quadratic action for
  proto--field fluctuations propagating on the corresponding effective metric.
  \item Using the resulting propagator, we compute the momentum--space
  two--point function \(\langle h_{\mu\nu}(k)\,h_{\rho\sigma}(-k)\rangle\) and
  decompose it into standard spin projectors.  We show that the
  transverse--traceless sector exhibits a massless spin--2 pole with positive
  residue, while scalar components are either massive or parametrically
  suppressed in the infrared.
  \item We relate the normalisation of the massless spin--2 pole to the
  effective Planck scale appearing in the induced Einstein--Hilbert term,
  thereby identifying the composite graviton of PFGM with the tensor mode of
  the induced gravitational effective action.
  \item In the window where the induced Einstein--Hilbert term dominates over
  higher--curvature corrections, we derive the linearised equations of motion
  for the transverse--traceless sector and show that they reduce to an
  Einstein--like wave equation sourced by the standard stress--energy tensor,
  with controlled higher--derivative corrections.
  \item We discuss degrees of freedom and emergent gauge structure, clarifying
  how the two physical tensor polarisations arise from the underlying scalar
  dynamics and under which conditions the effective diffeomorphism invariance
  is a good symmetry.
\end{itemize}

\subsection{Organisation of the paper}
\label{subsec:intro_organisation}

The remainder of the paper is organised as follows.
In \cref{sec:setup} we briefly review the microscopic proto--field action and
the disformal composite metric, and summarise the induced Einstein--Hilbert
term and its regime of validity.  In \cref{sec:backgrounds} we introduce the
backgrounds of interest and the fluctuation decomposition of the proto--field.
In \cref{sec:composite_fluctuations} we construct the composite metric
fluctuation operator and derive the quadratic action and propagator for the
proto--field.  In \cref{sec:spin2_propagator} we compute the composite
\(\langle h h\rangle\) correlator, perform the spin decomposition, and isolate
the massless spin--2 pole.  In \cref{sec:linearised_dynamics} we connect this
composite graviton to the induced Einstein--Hilbert action and derive the
linearised equations of motion and coupling to matter in the infrared.  In
\cref{sec:DOF_gauge} we discuss degrees of freedom and emergent gauge
structure.  Finally, in \cref{sec:discussion,sec:outlook} we place our results
in the broader context of emergent gravity scenarios and outline directions
for future work.

\section{Proto--field setup and composite metric}
\label{sec:setup}

In this section we briefly review the microscopic proto--field action, the
disformal composite metric to which matter is coupled, and the structure of the
induced gravitational effective action.  The emphasis is on fixing notation and
summarising those aspects of the setup that are directly relevant for the
emergent spin--2 sector studied in later sections.

\subsection{Microscopic proto--field action}
\label{subsec:setup_action}

The fundamental degree of freedom is a single real scalar field \(\Phi\)
propagating on a fixed Minkowski background \(\eta_{\mu\nu}\).  Its microscopic
action is of \(P(X)\) type, augmented by a stabilising quartic derivative term
that ensures the existence of a healthy band of backgrounds,
\begin{equation}
  S_\Phi = \int \mathrm{d}^4x\,
  \bigg[ P(X) + \frac{\beta_2}{\Lambda^4}
   (\partial_\mu\partial_\nu\Phi\,\partial^\mu\partial^\nu\Phi) \bigg],
  \qquad
  X \equiv -\frac12\,\eta^{\mu\nu}\partial_\mu\Phi\,\partial_\nu\Phi,
\end{equation}
where \(P(X)\) is a smooth function with suitable derivatives in the regime of
interest, \(\beta_2\) is a dimensionless coefficient, and \(\Lambda\) denotes a
microscopic scale controlling the onset of higher--derivative corrections.
The second term is written schematically; in practice only a particular
combination of quartic structures is needed to achieve stability, but its
detailed form will not be important in what follows.

We assume that there exists a ``healthy band'' of backgrounds for which small
fluctuations of \(\Phi\) are free of ghosts and gradient instabilities.  In
terms of the \(P(X)\) sector alone this typically requires
\begin{equation}
  P'(X_0) > 0,
  \qquad
  P'(X_0) + 2X_0 P''(X_0) > 0,
\end{equation}
where \(X_0\) is the value of \(X\) evaluated on the background solution
\(\Phi_0\).  The quartic term controlled by \(\beta_2\) can then be tuned to
ensure hyperbolicity and to avoid strong coupling in the fluctuation sector,
as discussed in more detail in the induced--gravity analysis.

\subsection{Disformal composite metric and effective geometry}
\label{subsec:setup_metric}

The effective geometry seen by matter is encoded in a rank--one disformal
composite metric built from the proto--field gradient,
\begin{equation}
  g^{\rm eff}_{\mu\nu}[\Phi]
  = \eta_{\mu\nu} + \alpha\,\partial_\mu\Phi\,\partial_\nu\Phi,
\end{equation}
with \(\alpha\) a constant parameter of mass dimension \(-4\).  This metric is
invertible as long as \(1 + \alpha\,\partial_\lambda\Phi\,\partial^\lambda\Phi
\neq 0\).  For later use it is convenient to write the inverse metric as
\begin{equation}
  g_{\rm eff}^{\mu\nu}
  = \eta^{\mu\nu}
  - \frac{\alpha}{1+\alpha\,(\partial\Phi)^2}
    \,\partial^\mu\Phi\,\partial^\nu\Phi,
  \qquad
  (\partial\Phi)^2 \equiv \eta^{\rho\sigma}\partial_\rho\Phi\,\partial_\sigma\Phi,
\end{equation}
and the determinant as
\begin{equation}
  \sqrt{-g_{\rm eff}} = \sqrt{-\eta}\,
  \big[1 + \alpha\,(\partial\Phi)^2\big]^{1/2},
\end{equation}
where \(\sqrt{-\eta}=1\) in Cartesian coordinates on Minkowski space.

Matter fields \(\Psi_{\rm m}\) are assumed to be minimally coupled to the
composite metric \(g^{\rm eff}_{\mu\nu}\),
\begin{equation}
  S_{\rm m}[g^{\rm eff},\Psi_{\rm m}]
  = \int \mathrm{d}^4x\,\sqrt{-g_{\rm eff}}\,
    \mathcal{L}_{\rm m}\big(g^{\rm eff}_{\mu\nu},\Psi_{\rm m}\big),
\end{equation}
so that their stress--energy tensor is defined in the usual way with respect to
\(g^{\rm eff}_{\mu\nu}\).  This choice ensures that any emergent tensor mode
constructed from \(g^{\rm eff}_{\mu\nu}\) will couple universally to matter in
the low--energy regime, provided the induced gravitational action for
\(g^{\rm eff}_{\mu\nu}\) takes the Einstein--Hilbert form.

\subsection{Induced Einstein--Hilbert term and IR effective theory}
\label{subsec:setup_induced_EH}

Integrating out proto--field fluctuations around a given background generates
an effective action for the composite metric \(g^{\rm eff}_{\mu\nu}\).  At
one--loop order, and within the healthy band described above, the leading
geometric terms take the schematic form
\begin{equation}
  S_{\rm grav}^{\rm eff}[g^{\rm eff}]
  = \int \mathrm{d}^4x\,\sqrt{-g_{\rm eff}}\,
    \bigg[ \frac{\Mpl^2}{2}\,R[g^{\rm eff}]
           + c_1 R^2[g^{\rm eff}] + c_2 C^2[g^{\rm eff}]
           + \cdots \bigg],
\end{equation}
where \(R\) is the Ricci scalar, \(C^2\) denotes the square of the Weyl tensor,
\(\Mpl\) is an induced Planck mass, and the ellipsis stands for higher--order
curvature invariants and nonlocal contributions suppressed at low energies.
The coefficients \(\Mpl^2,c_1,c_2,\ldots\) are calculable functions of the
microscopic parameters of the proto--field theory and of the background value
of \(X\).

In the regime where curvatures and momenta are small compared to the scale set
by \(|c_{1,2}|/\Mpl^2\), the Einstein--Hilbert term dominates and the
gravitational dynamics are well approximated by general relativity with
effective Planck mass \(\Mpl\).  Previous work has shown that, for suitable
choices of microscopic parameters, this induced--gravity effective theory
passes standard weak--field and post--Newtonian tests, and admits nontrivial
mesoscopic behaviour relevant to dark--matter phenomenology.  In the present
paper we focus on the microscopic origin of the tensor sector associated with
the induced Einstein--Hilbert term, and in particular on the identification of
a composite graviton operator built from the proto--field.

\section{Backgrounds and fluctuation decomposition}
\label{sec:backgrounds}
% - Choose simple backgrounds where disformal structure is nontrivial
% - Perform background-field split \Phi = \Phi_0 + \chi
% - Introduce canonical fluctuation \psi and its operator

\subsection{Constant-gradient background on Minkowski}
\label{subsec:bg_constant_gradient}
% - \Phi_0 = q_\mu x^\mu, q_\mu = const
% - Effective metric: g^{eff}_{\mu\nu}[\Phi_0] = \eta_{\mu\nu} + \alpha q_\mu q_\nu
% - Choice of frame q_\mu = (q,0,0,0) and properties

\subsection{Cosmological (FRW-like) background}
\label{subsec:bg_FRW}
% - Optional: \Phi_0(t) and homogeneous FRW metric
% - Comment on extension of results beyond Minkowski + grad

\subsection{Healthy band and stability conditions}
\label{subsec:bg_healthy_band}
% - Recall conditions for absence of ghosts and gradient instabilities
% - Constraints on P'(X), P''(X), quartic term, etc.


% =========================================================
\section{Composite metric fluctuations}
\label{sec:composite_fluctuations}
% - Define fluctuations of the composite metric in terms of \chi
% - Derive quadratic action for \chi (or canonical \psi)
% - Identify composite h_{\mu\nu}[\Phi] operator

\subsection{Background-field split and canonical fluctuation}
\label{subsec:fluct_split}
% - \Phi = \Phi_0 + \chi, relation to canonically normalised \psi
% - Quadratic expansion of microscopic action around \Phi_0
% - Kinetic operator and principal symbol

\subsection{Linearised composite metric}
\label{subsec:fluct_linear_metric}
% - Expand g^{eff}_{\mu\nu}[\Phi] to linear order in \chi:
%   h_{\mu\nu} \equiv \delta g^{eff}_{\mu\nu}[\chi]
% - Express h_{\mu\nu} in terms of q_\mu and \partial_\mu \chi on chosen background
% - Discuss symmetries and gauge redundancy (if any) at this stage

\subsection{Quadratic action and proto--field propagator}
\label{subsec:fluct_quadratic_action}
% - Write S^{(2)}[\chi;\Phi_0] (or \psi) on the background
% - Derive momentum-space propagator for \chi (or \psi)
% - Identify effective mass/dispersion and IR regime of interest


% =========================================================
\section{Emergent spin--2 propagator}
\label{sec:spin2_propagator}
% - Construct 2-pt function of h_{\mu\nu}
% - Decompose into spin projectors
% - Isolate massless spin--2 pole and discuss residue/sign

\subsection{Proto--field propagator on constant-gradient background}
\label{subsec:spin2_chi_prop}
% - Write \langle \chi(k) \chi(-k) \rangle
% - Comment on modifications vs pure Minkowski (if any)

\subsection{Composite \texorpdfstring{$hh$}{hh} correlator in momentum space}
\label{subsec:spin2_hh_correlator}
% - Define h_{\mu\nu}(x) from \chi and compute \langle h h \rangle
% - Leading Gaussian approximation and Wick contractions
% - Structure of tensor indices in momentum space

\subsection{Spin decomposition and projectors}
\label{subsec:spin2_projectors}
% - Introduce transverse projector \theta_{\mu\nu}
% - Define spin--2, spin--1, and spin--0 projectors \mathcal{P}^{(2)}, \mathcal{P}^{(1)}, ...
% - Decompose \langle h h \rangle into A(k^2)\mathcal{P}^{(2)} + B(k^2)\mathcal{P}^{(1)} + ...

\subsection{Massless spin--2 pole and residue}
\label{subsec:spin2_massless_pole}
% - Show A(k^2) \sim Z_2 / k^2 in the IR, with Z_2 > 0
% - Interpret Z_2 in terms of effective Planck mass / coupling
% - Discuss robustness under small deformations of background

\subsection{Scalar sector and absence of ghosts}
\label{subsec:spin2_scalar_sector}
% - Analyse B(k^2), C(k^2), D(k^2) for scalar/spin--1 pieces
% - Show no massless scalar ghost pole at k^2 = 0
% - Comment on massive scalar excitations and IR suppression

\subsection{Spectral representation and positivity}
\label{subsec:spin2_spectral}
% - Optional short, main-text summary of Källén--Lehmann representation
% - Presence of delta-function in \rho_2(\mu^2) at \mu^2 = 0
% - Positivity constraints; details in an appendix


% =========================================================
\section{Linearised dynamics and coupling to matter}
\label{sec:linearised_dynamics}
% - Connect composite h_{\mu\nu}[\Phi] to induced EH action
% - Derive linearised equations in IR
% - Show Einstein-like TT dynamics and standard coupling to T_{\mu\nu}

\subsection{Induced Einstein--Hilbert action and higher-curvature terms}
\label{subsec:lin_EH_R2}
% - Write effective action: EH + R^2 + C^2 + ...
% - Discuss IR window where EH dominates and higher-curvature terms are small

\subsection{Linearised equations and gauge choice}
\label{subsec:lin_equations}
% - Linearise around background g^{eff}_{\mu\nu} = \bar{g}_{\mu\nu}
% - Choose de Donder/harmonic gauge
% - Derive equation for h_{\mu\nu} in momentum space (including higher-curvature corrections)

\subsection{Transverse-traceless sector and wave equation}
\label{subsec:lin_TT_sector}
% - Project onto TT sector: h^{\rm TT}_{\mu\nu}
% - Show \Box h^{\rm TT}_{\mu\nu} \simeq 0 in IR
% - Estimate size of corrections from R^2/C^2 operators

\subsection{Coupling to matter and GR limit}
\label{subsec:lin_coupling_matter}
% - Introduce matter coupled to g^{eff}_{\mu\nu}
% - Show linear coupling h^{\mu\nu} T_{\mu\nu}
% - Recover GR-like relation in IR:
%   \Box h^{\rm TT}_{\mu\nu} \sim 8\pi G_{\mathrm{eff}} T^{\rm TT}_{\mu\nu}


% =========================================================
\section{Degrees of freedom and emergent gauge structure}
\label{sec:DOF_gauge}
% - Count DOFs in proto-field description vs emergent metric description
% - Clarify role of emergent diffeomorphism invariance at low energies

\subsection{Degrees of freedom in the proto--field description}
\label{subsec:DOF_proto}
% - Naive DOF counting for real scalar \Phi
% - Constraints from equations of motion and stability

\subsection{Mapping to metric degrees of freedom}
\label{subsec:DOF_metric}
% - Relation between \delta\Phi and \delta g^{eff}_{\mu\nu}
% - How many independent components of h_{\mu\nu}[\Phi] are actually dynamical?

\subsection{Emergent diffeomorphisms in the IR}
\label{subsec:DOF_emergent_diff}
% - Effective action is diffeo invariant in g^{eff}_{\mu\nu}
% - Gauge redundancy reduces metric DOFs to 2 TT modes
% - Interpretation in terms of composite graviton built from \Phi

\subsection{Background dependence and limitations}
\label{subsec:DOF_limitations}
% - Where does emergent gauge structure hold? (e.g. healthy band, near-flat backgrounds)
% - Limitations: strong gradients, beyond EFT cutoff, nonperturbative regimes


% =========================================================
\section{Discussion}
\label{sec:discussion}
% - Interpret main results qualitatively
% - Compare to other emergent gravity / induced gravity scenarios
% - Highlight novelty: explicit composite graviton operator with spin--2 pole

\subsection{Comparison with previous PFGM results}
\label{subsec:disc_PFGM}
% - Relation to Papers I--III (PN tests, strong-field structure)
% - Relation to cosmological constant solution and mesoscopic halos
% - How emergent spin--2 mode fits into the global PFGM picture

\subsection{Relation to other emergent gravity frameworks}
\label{subsec:disc_other_frameworks}
% - Brief comparison with Sakharov-like induced gravity, analogue gravity, etc.
% - Similarities and differences in microscopic degrees of freedom

\subsection{Conceptual implications}
\label{subsec:disc_conceptual}
% - What it means for the graviton to be composite
% - Comments on UV completion, unitarity, and locality


% =========================================================
\section{Outlook}
\label{sec:outlook}
% - Future directions and open questions

\subsection{Extensions to cosmology and inhomogeneous backgrounds}
\label{subsec:outlook_cosmo}
% - Generalisation to FRW, late-time cosmology
% - Impact on tensor modes / primordial gravitational waves

\subsection{Beyond linear order: interactions and non-Gaussianities}
\label{subsec:outlook_interactions}
% - Higher-point functions of h_{\mu\nu}[\Phi]
% - Self-interactions and departures from GR at nonlinear level

\subsection{Links to mesoscopic halos and dark matter phenomenology}
\label{subsec:outlook_mesoscopic}
% - Possible interplay between composite graviton sector and mesoscopic coherence scale
% - How this feeds into dark-matter/halo phenomenology in other PFGM papers

\subsection{Open problems}
\label{subsec:outlook_open_problems}
% - List of concrete technical and conceptual questions left for future work


% =========================================================
\appendix

\section{Conventions and spin projectors}
\label{app:conventions_projectors}
% - Metric signature, Fourier conventions
% - Definition of \theta_{\mu\nu}, \omega_{\mu\nu}
% - Explicit forms of \mathcal{P}^{(2)}, \mathcal{P}^{(1)}, \mathcal{P}^{(0-s)}, \mathcal{P}^{(0-w)}
% - Useful identities for projector algebra

\subsection{Fourier conventions and kinematics}
\label{app:conv_fourier}
% - Define Fourier transform and momentum-space measures

\subsection{Projector algebra}
\label{app:conv_proj_algebra}
% - Orthogonality and completeness relations among projectors

% ---------------------------------------------------------
\section{Quadratic action and proto--field propagator}
\label{app:quad_action}
% - Detailed expansion of P(X) + quartic term to second order
% - Canonical normalisation and dispersion relation

\subsection{Expansion around constant-gradient background}
\label{app:quad_const_grad}
% - Explicit expressions for coefficients in quadratic operator

\subsection{Dispersion relation and healthy band}
\label{app:quad_dispersion}
% - Derive dispersion relation and conditions for stability

% ---------------------------------------------------------
\section{Composite \texorpdfstring{$hh$}{hh} correlator: detailed derivation}
\label{app:hh_correlator}
% - Step-by-step calculation of \langle h_{\mu\nu}(k) h_{\rho\sigma}(-k) \rangle
% - Wick contractions, tensor structures, and symmetries

\subsection{Wick contractions and tensor structures}
\label{app:hh_wick}
% - Show how q_\mu, k_\mu, and \eta_{\mu\nu} enter the correlator

\subsection{Extraction of spin coefficients}
\label{app:hh_spin_coeffs}
% - Project onto \mathcal{P}^{(2)}, \mathcal{P}^{(1)}, \mathcal{P}^{(0)} to obtain A(k^2), B(k^2), C(k^2), D(k^2)

% ---------------------------------------------------------
\section{Spectral representation and positivity}
\label{app:spectral}
% - Källén--Lehmann representation for \langle h h \rangle
% - Identification of \rho_2(\mu^2) and \rho_0(\mu^2)

\subsection{Källén--Lehmann representation}
\label{app:spectral_KL}
% - General formulas for tensor operators

\subsection{Spectral densities and positivity bounds}
\label{app:spectral_bounds}
% - Show \rho_2(\mu^2) has a positive delta at \mu^2 = 0
% - Discuss constraints on \rho_0(\mu^2)

% ---------------------------------------------------------
\section{Alternative backgrounds and robustness checks}
\label{app:alt_backgrounds}
% - Explore small curvature / FRW background
% - Check that the spin--2 pole persists under mild deformations

\subsection{Perturbative curvature corrections}
\label{app:alt_curvature}
% - Treat curvature as small parameter; sketch impact on propagator

\subsection{Homogeneous time-dependent background}
\label{app:alt_FRW}
% - Briefly outline modifications in a FRW-like setup

% ---------------------------------------------------------
\section{Connection to induced-gravity EFT constraints}
\label{app:induced_constraints}
% - Summarise parameter constraints from previous PFGM/induced-EH analyses
% - Map them to allowed region for emergent spin--2 sector

\subsection{Bounds from Solar-System and binary tests}
\label{app:induced_bounds}
% - Recall relevant limits on coefficients of higher-curvature terms

\subsection{Implications for composite graviton parameters}
\label{app:induced_implications}
% - How these bounds translate to Z_2, G_{\mathrm{eff}}, etc.

% =========================================================
\bibliographystyle{unsrt}
\bibliography{PFGM-CompositeGraviton} % adjust .bib name as needed

\end{document}
